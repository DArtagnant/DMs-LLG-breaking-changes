% to change the appearance of the header, questions, problems or subproblems, see the homework.cls file or
% override the \Problem, \Subproblem, \question or \printtitle commands.

% The hidequestions option hides the questions. Remove it to print the questions in the text.

% CHANGER CE PATH S'IL EST DIFFERENT POUR VOUS
\documentclass[]{../templates/homework}
\usepackage[french]{babel}

\homeworksetup{
	username={Thomas Diot, Jim Garnier, Jules Charlier, Pierre Gallois \\ 1E1},
	course={Matière},
	setnumber=2}
\begin{document}
%%%%%%%%%%%%%%%%%%%%%%%%%%%
\problem*{1}{Équations fonctionnelles}
\textbf{Partie A.}

$\forall m, n \in \mathbb{N}, g(m + n) = g(m) \times g(n)$
\subproblem
On suppose qu'il existe une telle fonction $g$.
\question
Comme $n$ et $m$ appartiennent à $\mathbb{N}$, l'unique manière d'avoir $n + m = 0$ est que $n = 0$ et $m = 0$.
On a donc :
\begin{align}
	g(m + n) &= g(m) \times g(n) \notag \\
	g(0) &= g(0) \times g(0) \label{eq:g0=g0xg0}
\end{align}
Nous divisons deux cas :

\textit{Si} $g(0) = 0$ :
Alors nous avons bien $g(0) = 0$.

\textit{Si} $g(0) > 0$ :
Nous pouvons diviser \eqref{eq:g0=g0xg0} par $g(0)$ ce qui donne $g(0) = 1$.

Nous avons prouvé que $g(0) = 0$ ou $g(0) = 1$.


%%%%%%%%%%%%%%%%%%%%%%%%%%%
	\problem*{2}{Nombres Échangeables}
	\subproblem
	En prenant $a = 3, b=-2$, on a bien $f_{a,b}(2) = 3$ et $f_{a,b} = 3-1 = 2$.
	\subproblem
	
	Supposons que $(x,y) \in \mathbb R^2$ est échangeable. On exclut le cas trivial $(x,x)$ qui est évidemment échangeable, et pour lequel $|x-x| = 0 \leq 1$. Sans perte de généralité, ordonnons donc par la suite $x < y$. On a le système :
	\begin{equation*}
		\begin{cases}
			a - \sqrt{x+b} = y \\
			a - \sqrt{y+b} = x
		\end{cases}
	\end{equation*}
	
	En faisant la différence des deux lignes, on trouve que $\sqrt{y+b} - \sqrt{x+b} = y-x = |x-y|$. De plus, on a :
	\begin{equation*}
		\begin{split}
		(\sqrt{y+b} - \sqrt{x+b})(\sqrt{y+b} + \sqrt{x+b}) &=y-x = |x-y| \\
		 \iff \sqrt{y+b} - \sqrt{x+b} &= \frac {|x-y|}{\sqrt{y+b} + \sqrt{x+b}} 
		 \end{split}
	\end{equation*}
	
	Donc : $$\frac {|x-y|}{\sqrt{y+b} + \sqrt{x+b}} = |x-y|$$
	Et $\sqrt{y+b} + \sqrt{x+b} = 1$.
	
	Donc si $(x,y)$ est échangeable, alors il existe $b\in\R$ tel que $\sqrt{y+b} + \sqrt{x+b} = 1$.
	
	Avant de prouver l'énoncé de la question, considérons la fonction $F: [-x;+\infty[$ définie par $F(b) = \sqrt{y+b} + \sqrt{x+b}$. Celle-ci est croissante comme somme de deux fonctions croissantes (par croissance de $x\mapsto \sqrt x$). De plus, elle est continue sur son intervalle de définition, comme la somme de deux fonctions continues sur cet intervalle.

	Pour prouver maintenant que si $(x,y)$ est échangeable, alors $|x-y| \leq 1$, prouvons sa contraposée : si $|x-y| > 1$, alors $(x,y)$ n'est pas échangeable. On procède par l'absurde.
	
	Supposons que $(x,y)$ est échangable. Alors il existe $b\in\R$ tel que $F(b) = 1$. Le minimum de $F$ sur son intervalle de définition est atteint en $b=-x$, avec $F(-x) = \sqrt{y-x} = \sqrt{|x-y|}$. Mais $|x-y| > 1$ et $\sqrt{|x-y|} > 1$. Donc $F(b) > 1$ pour tout $b\in \mathcal D_F$ et $b$ n'existe pas, ce qui est une contradiction.
	
	\subproblem
	Prouvons que si $|x-y| \leq 1$, alors $(x,y)$ est échangeable.
	
	Dans ce cas, on peut trouver $b\in\R$ tels que $F(b) \leq 1$ et $F(b) > 1$. En effet, si $b=-x$, alors $F(-x) = \sqrt{y-x} \leq 1$ par hypothèse, et $F(2-x) = \sqrt{2+(y-x)} + \sqrt(2) > 1$ car $\sqrt{2} > 1$.
	
	Par le théorème des valeurs intermédiaires, comme $F$ est continue sur $\mathcal D_F$, il existe $b_0 \in \mathcal D_F$ tel que $F(b_0) = 1$. Donc on a:
	\begin{equation*}
		\sqrt{y+b_0} - \sqrt{x+b_0} = \frac{|x-y|}{F(b_0)} = |x-y| = y-x
	\end{equation*}
	En particulier, en posant $a_0 = \sqrt{x+b_0} + y$, on a : $$a_0 - \sqrt{x+b_0} = y$$
	Et :$$a_0 - \sqrt{y+b_0} = y - (\sqrt{y+b_0} - \sqrt{x+b_0}) = y - (y-x) = x$$
	Donc $(x,y)$ sont échangeables avec $a_0, b_0$.
\end{document}