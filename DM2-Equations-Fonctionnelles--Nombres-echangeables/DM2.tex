% to change the appearance of the header, questions, problems or subproblems, see the homework.cls file or
% override the \Problem, \Subproblem, \question or \printtitle commands.

% The hidequestions option hides the questions. Remove it to print the questions in the text.

% CHANGER CE PATH S'IL EST DIFFERENT POUR VOUS
\documentclass[]{../templates/homework}
\usepackage[french]{babel}

\homeworksetup{
	username={Thomas Diot, Jim Garnier, Jules Charlier, Pierre Gallois \\ 1E1},
	course={Matière},
	setnumber=2}
\begin{document}
%%%%%%%%%%%%%%%%%%%%%%%%%%%
\problem*{1}{Équations fonctionnelles}
\textbf{Partie A.}

\subproblem
\textit{Analyse.}

On suppose qu'il existe une telle fonction $g$.
\question \\
Soient $n$ et $m$ appartenant à $\mathbb{N}$. L'unique manière d'obtenir $n + m = 0$ est que $n = 0$ et $m = 0$.
Cela donne :
\begin{align*}
	g(m + n) & = g(m) \times g(n) \\
	g(0)     & = g(0)^2
\end{align*}
Les solutions de cette équation sont $\{0, 1\}$.

Nous avons prouvé que $g(0) = 0$ ou $g(0) = 1$.

\question \\
Soit $n$ dans $\mathbb{N}$,
\begin{align*}
	g(n) & = g(n + 0) \\
	& = g(n) \times g(0) \\
	& = 0 \quad \text{Puisque on est dans le cas où $g(0) = 0$}
\end{align*}
Si $g(0) = 0$, alors $g$ est la fonction nulle.

\question \\
Nous raisonnons par récurrence. On suppose que $g(0) = 1$ et $g(n) = g(1)^n$.

\textit{Initialisation :}
Par hypothèse, $g(0) = 1$.

\textit{Hérédité :}
\begin{align*}
	g(n + 1) &= g(n) \times g(1) \\
	&= g(1)^n \times g(1) \quad \text{D'après la supposition de départ.} \\
	&= g(1)^{(n + 1)}
\end{align*}

Nous avons prouvé par récurrence que pour tout $n \in \mathbb{N^*}$, $g(n) = a^n$ où $a=g(1)$.

\subproblem
\textit{Synthèse.}

Nous avons deux candidats :
\begin{itemize}
	\item La fonction nulle : $\forall n \in \mathbb{N}, g(n) = 0$
	\item Une fonction puissance : $\forall n \in \mathbb{N}, g(n) = g(1)^n$
\end{itemize}
Nous devons vérifier la condition : $\forall m, n \in \mathbb{N}, g(m + n) = g(m) \times g(n)$.

Vérifions pour la fonction nulle :

Pour tous $m, n \in \mathbb{N}$ :
\begin{align*}
	g(m + n) &= 0 \\
	&= 0 \times 0 \\
	&= g(m) \times g(n)
\end{align*}
La condition est bien vérifiée.

Vérifions pour la fonction puissance :
Pour tous $m, n \in \mathbb{N}$ :
\begin{align*}
	g(m + n) &= g(1)^{(m + n)} \\
	&= g(1)^m \times g(1)^n \\
	&= g(m) \times g(n)
\end{align*}
La condition est bien vérifiée.

La fonction nulle et $g(n) = g(1)^n$ sont donc les deux seules fonctions qui respectent $\forall m, n \in \mathbb{N}, g(m + n) = g(m) \times g(n)$.
%%%%%%%%%%%%%%%%%%%%%%%%%%%
\problem*{2}{Nombres Échangeables}
\subproblem
En prenant $a = 3, b=-2$, on a bien $f_{a,b}(2) = 3$ et $f_{a,b} = 3-1 = 2$.
\subproblem

Supposons que $(x,y) \in \mathbb R^2$ est échangeable. On exclut le cas trivial $(x,x)$ qui est évidemment échangeable, et pour lequel $|x-x| = 0 \leq 1$. Sans perte de généralité, ordonnons donc par la suite $x < y$. On a le système :
\begin{equation*}
	\begin{cases}
		a - \sqrt{x+b} = y \\
		a - \sqrt{y+b} = x
	\end{cases}
\end{equation*}

En faisant la différence des deux lignes, on trouve que $\sqrt{y+b} - \sqrt{x+b} = y-x = |x-y|$. De plus, on a :
\begin{equation*}
	\begin{split}
		(\sqrt{y+b} - \sqrt{x+b})(\sqrt{y+b} + \sqrt{x+b}) & =y-x = |x-y|                             \\
		\iff \sqrt{y+b} - \sqrt{x+b}                       & = \frac {|x-y|}{\sqrt{y+b} + \sqrt{x+b}}
	\end{split}
\end{equation*}

Donc : $$\frac {|x-y|}{\sqrt{y+b} + \sqrt{x+b}} = |x-y|$$
Et $\sqrt{y+b} + \sqrt{x+b} = 1$.

Donc si $(x,y)$ est échangeable, alors il existe $b\in\R$ tel que $\sqrt{y+b} + \sqrt{x+b} = 1$.

Avant de prouver l'énoncé de la question, considérons la fonction $F: [-x;+\infty[$ définie par $F(b) = \sqrt{y+b} + \sqrt{x+b}$. Celle-ci est croissante comme somme de deux fonctions croissantes (par croissance de $x\mapsto \sqrt x$). De plus, elle est continue sur son intervalle de définition, comme la somme de deux fonctions continues sur cet intervalle.

Pour prouver maintenant que si $(x,y)$ est échangeable, alors $|x-y| \leq 1$, prouvons sa contraposée : si $|x-y| > 1$, alors $(x,y)$ n'est pas échangeable. On procède par l'absurde.

Supposons que $(x,y)$ est échangable. Alors il existe $b\in\R$ tel que $F(b) = 1$. Le minimum de $F$ sur son intervalle de définition est atteint en $b=-x$, avec $F(-x) = \sqrt{y-x} = \sqrt{|x-y|}$. Mais $|x-y| > 1$ et $\sqrt{|x-y|} > 1$. Donc $F(b) > 1$ pour tout $b\in \mathcal D_F$ et $b$ n'existe pas, ce qui est une contradiction.

\subproblem
Prouvons que si $|x-y| \leq 1$, alors $(x,y)$ est échangeable. On suppose toujours que $x \leq y$.

Dans ce cas, on peut trouver $b\in\R$ tels que $F(b) \leq 1$ et $F(b) > 1$. En effet, si $b=-x$, alors $F(-x) = \sqrt{y-x} \leq 1$ par hypothèse, et $F(2-x) = \sqrt{2+(y-x)} + \sqrt(2) > 1$ car $\sqrt{2} > 1$.

Par le théorème des valeurs intermédiaires, comme $F$ est continue sur $\mathcal D_F$, il existe $b_0 \in \mathcal D_F$ tel que $F(b_0) = 1$. Ainsi, on a:
\begin{equation*}
	\sqrt{y+b_0} - \sqrt{x+b_0} = \frac{|x-y|}{F(b_0)} = |x-y| = y-x
\end{equation*}
En particulier, en posant $a_0 = \sqrt{x+b_0} + y$, on a : $$a_0 - \sqrt{x+b_0} = y$$
Et :$$a_0 - \sqrt{y+b_0} = y - (\sqrt{y+b_0} - \sqrt{x+b_0}) = y - (y-x) = x$$
Donc $(x,y)$ sont échangeables avec $a_0, b_0$.
\end{document}