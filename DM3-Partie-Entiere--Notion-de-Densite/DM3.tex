% to change the appearance of the header, questions, problems or subproblems, see the homework.cls file or
% override the \Problem, \Subproblem, \question or \printtitle commands.

% The hidequestions option hides the questions. Remove it to print the questions in the text.

% CHANGER CE PATH S'IL EST DIFFERENT POUR VOUS
\documentclass[]{../templates/homework}
\usepackage[french]{babel}

% Pour le thème des scripts
\usepackage[scaled]{inconsolata}
\usepackage{listings}
\usepackage{xcolor}
\usepackage{tcolorbox}
\definecolor{vscode-bg}{HTML}{1E1E1E} 
\definecolor{vscode-keyword}{HTML}{569CD6}
\definecolor{vscode-comment}{HTML}{6A9955}
\definecolor{vscode-string}{HTML}{CE9178}
\definecolor{vscode-function}{HTML}{DCDCAA}
\lstset{
    language=Python,
    backgroundcolor=\color{vscode-bg},
    basicstyle=\ttfamily\color{white}\selectfont,
    keywordstyle=\color{vscode-keyword}\bfseries,
    commentstyle=\color{vscode-comment}\itshape,
    stringstyle=\color{vscode-string},
    identifierstyle=\color{vscode-function},
    showstringspaces=false,
    numbers=left,
    numberstyle=\ttfamily\tiny\color{gray},
    frame=none,
    breaklines=true,
    tabsize=4,
    columns=fullflexible,
    keepspaces=false,
	numbersep=10pt,
}
\newtcolorbox{vscodebox}{
    sharp corners=southwest,
    colback=vscode-bg!5!vscode-bg,
    colframe=vscode-bg,
    boxrule=2mm,
    rounded corners=southwest,
    left=17pt,
    right=17pt
}



\homeworksetup{
	username={Thomas Diot, Jim Garnier, Jules Charlier, Pierre Gallois \\ 1E1},
	course={Mathématiques},
	setnumber=3
}

	
\begin{document}

\problem*{1}{Partie entière}

\subproblem

$\lfloor\frac{5}{2}\rfloor = 2$

$\lfloor-\pi\rfloor = -4$

$\lfloor\frac{2\pi}{7}\rfloor = 0$

\hfill

Version originale :
\begin{vscodebox}
\begin{lstlisting}
def partent(x):
	n = 0
	# Quand x est negatif, cette condition est fausse des le depart
	while n+1 <= x:
		n += 1
	return n

from math import pi
print(partent(5/2)) # 2
print(partent(-pi)) # 0 FAUX
print(partent(2*pi/7)) # 0
\end{lstlisting}
\end{vscodebox}

\hfill

Version corrigée :

\begin{vscodebox}
\begin{lstlisting}
def partent(x):
	n = 0
	while n+1 <= abs(x):
		n += 1
	if x >= 0:
		return n
	else:
		return -n-1

from math import pi
print(partent(5/2)) # 2
print(partent(-pi)) # -4
print(partent(2*pi/7)) # 0
\end{lstlisting}
\end{vscodebox}

\hfill

Versions optimisées :

\begin{vscodebox}
\begin{lstlisting}
def partent1(x):
	n = 0
	abs_x = abs(x)
	while (n := n+1) <= abs_x: pass
	return n-1 if x >= 0 else -n

def partent2(x):
	return int(x) - (1 if x <= 0 else 0)
\end{lstlisting}
\end{vscodebox}

\end{document}